\label{sec:results}
Dans cette section nous présentons les résultats de notre étude. Nous présentons
d'abord les résultats relative à la détection des odeurs de conception dans les
programmes étudiés. Puis, nous présentons les résultats relative de l'analyse des
résultats.
\todo{To be completed}

\subsection{Détection des odeurs de conception}
\label{sec:results1}
Ici nous présenterons les chiffres sur la détection des odeurs de conception
dans l'ensemble des repositories tel que la répartition des odeurs de
conception dans le dataset (Le nombre de repositories par odeurs de conception),
ou la profondeur (Nombre d'occurence d'une odeur de conception dans un
repository).


Cela soutiendra l'hypothèse selon laquelle les odeurs de conception peuvent être
détectées sur les programmes CNN à travers un model \emph{FAST} dans Pharo. Et de ce
fait, on peut donc détecter les odeurs de conception à l'aide de la modélisation.




\subsection{Analyse des résultats de la détection}
\label{sec:results2}
Ici nous présenterons les chiffres sur l'analyse des résultats de la détection
des odeurs de conception dans les programmes étudiés. Le but est de présenter le
classement des odeurs de conception, et de présenter les chiffres sur les
possibles correlation entre elles.

Cela soutiendra l'hypothèse selon laquelle il y a des odeurs de conception qui sont plus répandu que d'autres dans
les programmes CNN (dans le référentiel de notre étude) et que certaines odeurs
de conception présentent des correlations entre elles.

