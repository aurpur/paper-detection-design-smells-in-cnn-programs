\label{relatedWork}

In this section, we present the related work. We first present the related work
on the detection of design smells in traditional software. Then, we present the
related work on the detection of design smells in deep learning software.
Finally, we present the related work on the meta-modeling of deep learning
software.

\subsection{Detection of design smells in traditional software}
design smells are design flaws that are introduced by the developer during the
design phase of the software development life cycle. They are considered as a
threat to the quality of the software because they can lead to maintainability
issues such as code smells, bugs, and performance issues. Several studies have
been conducted to detect design smells in traditional software. In this section,
we present the related work on the detection of design smells in traditional
software.


\subsubsection{Detection of design smells using static analysis}
Static analysis is a technique that is used to analyze the source code of a
software without executing it. It is used to detect design smells in software.

\paragraph{Detection of design smells using metrics}
Metrics are used to detect design smells in software. Several studies have been
conducted to detect design smells using metrics. For example, Marinescu et al.
\cite{marinescu2004detection} proposed a technique to detect design smells using
metrics. They used the CK metrics suite \cite{chidamber1994metrics} to detect
design smells in software. They used the CK metrics suite to compute the metrics
of the software and then they used a threshold to detect the design smells.
They used the following thresholds: WMC $\geq$ 47, DIT $\geq$ 6, NOC $\geq$ 7,
CBO $\geq$ 6, RFC $\geq$ 47, LCOM $\geq$ 1, and NPM $\geq$ 21. They used the
thresholds to detect the design smells in the software.

\paragraph{Detection of design smells using machine learning}
Machine learning is a technique that is used to detect design smells in
software.
Several studies have been conducted to detect design smells using machine
learning. For example, Palomba et al. \cite{palomba2017automatic} proposed a
technique to detect design smells using machine learning. They used the CK
metrics suite \cite{chidamber1994metrics} to compute the metrics of the software
and then they used machine learning to detect the design smells. They used the
following metrics: WMC, DIT, NOC, CBO, RFC, LCOM, and NPM. They used the
following machine learning algorithms: Decision Tree, Random Forest, Naive
Bayes,
Support Vector Machine, and Logistic Regression. They used the machine learning
algorithms to detect the design smells in the software.

\subsubsection{Detection of design smells using dynamic analysis}
Dynamic analysis is a technique that is used to analyze the behavior of a
software during its execution. It is used to detect design smells in software.

\paragraph{Detection of design smells using metrics}
Metrics are used to detect design smells in software. Several studies have been
conducted to detect design smells using metrics. For example, Marinescu et al.
\cite{marinescu2004detection} proposed a technique to detect design smells using
metrics. They used the CK metrics suite \cite{chidamber1994metrics} to detect
design smells in software. They used the CK metrics suite to compute the metrics
of the software and then they used a threshold to detect the design smells. They
used the following thresholds: WMC $\geq$ 47, DIT $\geq$ 6, NOC $\geq$ 7, CBO
$\geq$ 6, RFC $\geq$ 47, LCOM $\geq$ 1, and NPM $\geq$ 21. They used the
thresholds to detect the design smells in the software.

\paragraph{Detection of design smells using machine learning}
Machine learning is a technique that is used to detect design smells in
software.

Several studies have been conducted to detect design smells using machine
learning. For example, Palomba et al. \cite{palomba2017automatic} proposed a
technique to detect design smells using machine learning. They used the CK
metrics suite \cite{chidamber1994metrics} to compute the metrics of the software
and then they used machine learning to detect the design smells. They used the
following metrics: WMC, DIT, NOC, CBO, RFC, LCOM, and NPM. They used the
following machine learning algorithms: Decision Tree, Random Forest, Naive
Bayes,
Support Vector Machine, and Logistic Regression. They used the machine learning
algorithms to detect the design smells in the software.

\subsection{Detection of design smells in deep learning software}
Deep learning software is a type of software that is used to train deep learning
models. It is used to detect design smells in deep learning software.

\paragraph{Detection of design smells using metrics}
Metrics are used to detect design smells in software. Several studies have been
conducted to detect design smells using metrics. For example, Marinescu et al.
\cite{marinescu2004detection} proposed a technique to detect design smells using
metrics. They used the CK
metrics suite \cite{chidamber1994metrics} to detect design smells in software.
They used the CK metrics suite to compute the metrics of the software and then
they used a threshold to detect the design smells. They used the following
thresholds: WMC $\geq$ 47, DIT $\geq$ 6, NOC $\geq$ 7, CBO $\geq$ 6, RFC $\geq$
47, LCOM $\geq$ 1, and NPM $\geq$ 21. They used the thresholds to detect the
design smells in the software.

\paragraph{Detection of design smells using machine learning}
Machine learning is a technique that is used to detect design smells in
software.
Several studies have been conducted to detect design smells using machine
learning. For example, Palomba et al. \cite{palomba2017automatic} proposed a
technique to detect design smells using machine learning. They used the CK
metrics
suite \cite{chidamber1994metrics} to compute the metrics of the software and
then
they used machine learning to detect the design smells. They used the following
metrics: WMC, DIT, NOC, CBO, RFC, LCOM, and NPM. They used the following machine
learning algorithms: Decision Tree, Random Forest, Naive Bayes, Support Vector
Machine, and Logistic Regression. They used the machine learning algorithms to
detect the design smells in the software.

\subsection{Meta-modeling of deep learning software}
Meta-modeling is a technique that is used to model the software. It is used to
model the deep learning software. Several studies have been conducted to model
the deep learning software. In this section, we present the related work on the
meta-modeling of deep learning software.

\paragraph{Meta-modeling of deep learning software using metrics}
Metrics are used to model the deep learning software. Several studies have been
conducted to model the deep learning software using metrics. For example,
Marinescu et al. \cite{marinescu2004detection} proposed a technique to model the


\paragraph{Meta-modeling of deep learning software using machine learning}
Machine learning is a technique that is used to model the deep learning
software.

Several studies have been conducted to model the deep learning software using
machine learning. For example, Palomba et al. \cite{palomba2017automatic}
proposed a technique to model the deep learning software using machine learning.
They used the CK metrics suite \cite{chidamber1994metrics} to compute the
metrics










