\label{sec:introduction}

L'apprentissage profond est un sous ensemble du machine learning qui utilise des
réseaux de neurones artificiels pour apprendre des données non structurées.
L'apprentissage profond est devenu très populaire ces dernières années grâce à
ses performances dans plusieurs domaines comme la reconnaissance d'image, la
reconnaissance vocale, la traduction automatique, etc. Les réseaux de neurones
artificiels sont des modèles mathématiques qui sont inspirés du fonctionnement
du cerveau humain. Ils sont composés de plusieurs couches de neurones qui sont
connectés entre eux. Chaque neurone est composé d'une fonction d'activation qui
permet de calculer la sortie du neurone en fonction de ses entrées. Il existe
plusieurs types d'architecture de réseaux de neurones artificiels. Ses
architecture ont été développées pour répondre à des problèmes spécifiques tels
que l'architecture de convolution pour la reconnaissance d'image, les réseaux de
neurones récurrents pour la reconnaissance vocale, etc. Dans ce papier, nous
nous intéressons à l'architecture de convolution car elle permet de traiter un
large ensemble de problèmes comme la reconnaissance d'image, la reconnaissance
de texte, la reconnaissance de séquence, etc. Par ailleurs l'architecture de
convolution fait partie des architectures feed-forward qui sont une classe de
réseaux de neurones artificiels qui sont composés de plusieurs couches de
neurones  et qui ne contiennent pas de cycles. Les réseaux de neurones
feed-forward sont les plus utilisés dans l'apprentissage profond.\\

Nous parlerons dans ce papier de programmes d'apprentissage profond pour
désigner les programmes contenant les réseaux de neurones. Tout comme les programmes
standard, les programmes d'apprentissage profond peuvent contenir des défauts.
Nous nous intéressons dans ce papier aux défauts de conception car se sont des
défauts introduits tôt dans le cycle de développement du logiciel et ils peuvent
avoir un impact négatif conséquent sur la performance et la qualité du logiciel. En effet les défauts de
conception sont des défauts qui sont introduits par le développeur lors de la
phase de  conception du logiciel.\\


% Défauts de conception
- Définition des défauts de conception\\
- Définit les défauts de conception dans les programmes d'apprentissage
profond\\
- Définition différence entre odeurs, bugs, erreurs et défauts\\


% Méta modélisation
Méta modélisation des programmes d'apprentissage profond\\


% Questions de recherche
- [RQ1] Comment détecter les défauts de conception dans les programmes d'apprentissage
profond?\\ % La technique de détection
- [RQ2] Quels sont les défauts de conception les plus répandu dans les programmes
d'apprentissage profond CNN?\\ % La répartition
- [RQ3] Exist-il des lien entre les défauts de conception dans les programmes? % Les relations 


