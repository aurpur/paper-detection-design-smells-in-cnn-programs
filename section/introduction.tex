\label{sec:introduction}

\Aurel{Définitions : Smart contract
Refactoring de code
Gas price Ethereum}

\Aurel{Orgainisation du papier}

Refactoring is the process of modifying and improving a code without changing its external behavior \cite{fowler2018refactoring} \cite{silva2016we} \cite{opdyke1992refactoring}. From this definition follows a practice that can be found in software engineering depending on the programming language.[x] utilise des projets Java, [x] quant à lui utilise le javascript ou encore [x] avec ses travaux sur le refactoring sur les logiciels mobiles. Par ailleurs, la motivation derrière le refactoring est beaucoup étudier dans la littérature, [x] et [x] se basent sur des sondage auprès des développeurs pour comprendre les motivations, [x] complête ses précédents papiers en ajoutant en plus d'une enquête, une analyse de 124 projets dans Github. Plusieurs études sont fait pour étudier l'impacte du refactoring sur la qualité du logiciel. \Aurel{Ajoute des exemples de travaux}.
Dans ce papier, nous nous intéressant de l'impacte des refactorings sur la performance d'un code en Solidity.\Aurel{Des travaux sur le refactoring en solodity} 

The remainder of the paper is organized as follows. Section \ref{sec:Analysis Methodology} describes our study method. We report our case study results in Section \ref{sec:results} followed by discussion of the findings in Section \ref{discussion}. Next, we describe the threats to validity in Section \ref{threats} and related work in Section \ref{related}. Finally, we conclude the paper in Section \ref{conc}. 


