\label{sec:threats}
Notre étude est sujet de plusieurs menaces à la validité. Dans cette section
nous présonterons ces menaces et les stratégies que nous avons utilisé pour les
mitiger.\\

Premièrement, notre étude est sujet de la menace de validité de construction. En
effet le modèle FAST généré par notre approche pourrait ne pas être correct et
le moèle pourrai ne ma être conforme au méta-modèle défini ou encore mal
représenter le programme d'apprentissage profond en entrée. Pour mitiger cette
menace, nous avons mise en place des tests unitaires pour chaque méthode
visiteur. Ces tests unitaires nous permettent de vérifier que les méthodes
implémentées dans l'importateur son correctement implémentées et que le modèle
FAST généré est conforme au méta-modèle défini. De plus, nous avons implémenté
pour un ensemble de données de tests (des exemples synthétiques) des tests
fonctionnels pour vérifier que le modèle FAST généré représente bien le
programme d'apprentissage profond en entrée.\\

Deuxièmement, notre étude est sujet de la menace de validité interne. Il
subsiste un risque que les conclusions obtenues ne soient pas causées par les
programmes en entrée. Pour mitiger cette menace, nous avons en plus de tester
notre système sur des exemples synthétiques, nous avons utilisé des projets
réels collectés dans Github pour valider le fonctionnement de notre système.\\

Troisièmement, notre étude est sujet de la menace de validité externe. Il y a un
risque que les résultats obtenus ne soient pas généralisables. Pour mitiger
cette menace, nous avons collecté aléatoirement des projets dans Github afin de
représenter différente application de l'architecture CNN.\\

Quatrièmement, notre étude est sujet de la menace de validité de fiabilité. Il y
a enfin un risque que les résultats obtenus ne soient pas reproductibles. Pour
mitiger cette menace, nous avons en plus d'utilisé des projets de différentes
applications de l'architecture CNN, utilisé des projets implémentées dans les
trois librairies les plus utilisées dans l'industrie (Tensorflow, Pytorch et Keras).



















% This section discusses threats to the validity of our study following the guidelines for case study research.
% \subsection{Threats to construct validity}



% \subsection{Threats to internal validity}




% \subsection{Threats to External validity}


% \subsection{Threats to reliability validity}