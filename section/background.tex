\label{sec:background}
In this section, we present previous work on detecting code smells in software.\\

The idea of detecting smells in software source code is a much-studied topic in
the literature. Indeed, several works have been proposing detection techniques
for more than a decade. Code smells have been studied from several angles, such
as their presence in software developed in a specific programming language
\cite{saboury2017empirical}, in the security of Infrastructure as Code source
code \cite{rahman2021security}, in microservices \cite{cernymicroservice}, or
their impact on the performance of Android applications
\cite{hecht2016empirical}. Fard et al \cite{6648192} present JSNOSE (precision:
93\% and average recall: 98\%) a technique for detecting code smells in the
JavaScript language. A combination of static and dynamic analysis to detect
client-side code smells. Chen et al. \cite{chen2016detecting} propose Pysmell, a
tool capable of detecting Python code smells with an average precision of 97.7\%.
Code smell detection has an impact on software quality, with Van Emden et al.
\cite{1173068} doing Java code quality assurance using code smell detection and
visualization. To cite just one example, Moha et al. \cite{moha2010decor}
propose a code smell detection and specification approach called DECOR with a
precision of 60.5\%, and a recall of 100\%. There are also modeling-based
approaches to smell detection in the literature, such as the approach proposed
by Araujo et al. \cite{araujo2018metamodeling} to identify organizational smells
in multi-agent systems via metamodeling, or that of Khomh et al.
\cite{khomh2009bayesian} who convert detection rules in the literature into a
probabilistic model to detect code smells. The latter leads the way to the
detection of code smells with the use of artificial intelligence. In recent
years, several works have proposed artificial intelligence-based approaches to
code smell detection \cite{alawadi2023fedcsd, sandouka2023python, alazba2023deep}. In
light of these works, we asked ourselves the following questions: \textbf{Is it
    possible to efficiently and statically detect code smells in deep learning
    software?} \\In the literature, Nikanjam et al. have initiated the answer to
these questions by proposing NeuraLint (precision: 100\% and recall: 70,5\%)
\cite{nikanjam2021automatic}, a model-based approach to code smell detection.
This approach uses modeling to achieve its goal. However, NeuraLint's scope was
limited to the feed-forward multilayer perceptron (MLP) architectures. Deep
learning is a growing field, and neural network architectures are becoming
increasingly complex. It is therefore necessary to be able to detect code smells
from other neural network architectures. The CNN architecture being one of the
most widely used in practice, we aim in this paper to allow the detection of
code smells in CNN-type neural networks.










