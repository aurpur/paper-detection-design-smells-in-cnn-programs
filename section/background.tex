\label{sec:background}
Dans cette section nous présenterons les précédents travaux sur la détection des
ordeurs de code dans les logiciels.\\

L'idée de détecter des ordeurs dans les codes sources des logiciels est un
sujet beaucoup étudié dans la littérature. En effet, plusieurs travaux proposent
des techniques de détection depuis plus d'une décénie. Les odeurs de code sont
étudiées selon plusieurs axes, comme par exemple leur présence dans des
logiciels développés dans un langage de programmation spécifique
\cite{saboury2017empirical}, la sécurité dans le code source d'Infrastructure as Code
\cite{rahman2021security}, dans les microservices \cite{cernymicroservice}, ou
leur impact sur la performance des applications Android
\cite{hecht2016empirical}.
Fard et al. \cite{6648192} propsent JSNOSE (precision: 93\% and average recall: 98\%) une technique de détection des
ordeurs de code du langage JavaScript. Une combinaise d'analyses statique et
dynamique pour détecter des odeurs de code coté client. Chen et al.
\cite{chen2016detecting} proposent quant à eux Pysmell, un outil capable de
détecter des odeurs de code du langage Python avec une précision moyen de
97.7\%. La détection d'odeurs de code a un impact sur la qualité du logiciel,
Van Emden et al. \cite{1173068} font de l'assurance qualité de code Java via la
détection et la visualisation d'odeurs de code. Pour ne citer que ces exemples,
Moha et al. \cite{moha2010decor} proposent une approche de spéccification et de
détection d'odeurs de code nommée DECOR avec une precision de 60.5\%,
et un recall de 100\%. On trouve également dans la litterature des approches de
détection d'odeurs basées sur la modélisation, comme l'approche proposée par
Araujo et al. \cite{araujo2018metamodeling} pour identifier les odeurs
organisationnelles dans les système multi-agents via la métamodélisation, ou
celle de Khomh et al. \cite{khomh2009bayesian} qui convertissent les règles de
détection dans la litterature en un modèle probabilistique dans le but de
détecter les odeurs de code. Ce dernier ouvre la voie à la détection d'odeurs
de code avec
l'aide de l'intelligence artificielle. On trouve de ce fait, dans les dernières
années plusieurs travaux proposant des approches de détection d'odeurs de code
basées sur l'intelligence artificielle \cite{alawadi2023fedcsd,
    sandouka2023python, alazba2023deep}. À la lumière de ces travaux, nous nous
sommes posés les questions suivantes: \textbf{Est-il possible de détecter
    efficacement et de
    manière statique les
    odeurs de code dans les logiciels d'apprentissage profond?} \\ Dans la litterature Nikanjam et al. ont
initer la réponse à ces questions en proposant NeuraLint (precision: 100\% and
recall: 70,5\%) \cite{nikanjam2021automatic}, une approche basées sur modèle
de détection d'odeurs de code. Cette approche utilise la métamodelisation pour atteindre son
but. Cependant la porté de NeuraLint ne s'est limité qu'aux architectures
feedforward multilayer perceptron (MLP). L'apprentissage profond est un domaine
en pleine expansion, et les architectures de réseaux de neurones sont de plus en
plus complexes. Il est donc nécessaire de pouvoir détecter des odeurs de code
d'autres architectures de réseaux de neurones. L'architecture CNN étant une des
plus utilisée dans la pratique, nous nous sommes fixés dans ce papier, l'objectif de
permettre la détection d'odeurs de code dans les réseaux de neurones de type
CNN.










