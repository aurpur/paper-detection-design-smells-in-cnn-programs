\label{sec:limitationsFutureWork}

\subsubsection{Limitations}
\label{sec:limitations}
Notre étude s'est fait dans un contexte limité que nous allons décrire dans
cette section. Quand nous parlons de programmes d'apprentissage profond, nous faisons référence
aux programmes contenant un réseau de neurones à convolution (CNN). Il est
donc important de noter qu'il existe d'autres types de programmes
d'apprentissage profond qui ne sont pas couverts par notre étude. De plus, il
existe plusieurs types de librairies qui permettent de créer des programmes
d'apprentissage profond. Notre étude se limite aux programmes créés à
partir des librairies TensorFlow, Keras et PyTorch. Et nos données proviennent
exclusivement de référentiel open source Github ou StackOverflow (pour les
exemples). Enfin, notre système de détection d'odeurs de conception est limité à
la détection des 8 ordeurs de conception décrit plus haut.\\

\subsubsection{Future Work}
\label{sec:futureWork}
Les travaux futurs peuvent aller dans plusieurs directions. En effet il est
possible d'appliquer notre approche à d'autres types de programmes
d'apprentissage profond, d'autres librairies, d'autres odeurs de conception,
d'autres langages de programmation et d'autres types de référentiels (non open
source par exemple).
Nous avons fait, une analyse statique des programmes d'apprentissage profond, il
est également envisageable de faire une analyse dynamique en simulant
l'exécution des programmes à travers de la modélisation. En plus de la
détection via la modélisation, d'autres champ de recherche sont la
détection via l'apprentissage automatique et la correction automatique des
odeurs de conception. Ces derniers sont des sujets de recherche très prométeurs
qui pourront avoir un impact très important sur la qualité des programmes
d'apprentissage profond futur.






