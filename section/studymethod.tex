\label{sec:study design}
In this sub-section, we describe the details of the data collection and processing approach followed to answer our different research questions.





\subsection{Data Collection}
\label{sec:data collection}
Le développement du système de détection d'odeurs de conception s'est fait à
partir des exemples de code de programmes d'apprentissage profond. Ces exemples
représentent des programmes d'apprentissage profond dans lesquels on y trouve
des odeurs de conceptions. Ses exemples ont été collectés par Amin et al. dans
le cadre de leur études empiriques sur les odeurs de conception dans les
programmes d'apprentissage profond. En effet, dans le cadre de cette étude, ils
ont présenté six exemples de programmes d'apprentissage profond présentant les
ordeurs de conceptions énuméré dans l'introduction \ref{sec:introduction}. Ces
exemples de code source ont été collectés à partir des plateformes StackOverflow
et Github.\\\\

Le système de détection d'odeurs de conception ainsi développé a servi sur un
ensemble de programmes d'apprentissage profond collectés à partir de la
plateforme Github. Ces programmes d'apprentissage profond ont été collectés
selon un processus autimatiques. En effet, nous avons utilisé l'API de Github et
précisement les requêtes de recherche de type \emph{search code}. Cette requête
permet de rechercher des fichiers dans les dépôts Github contenant des mots clés
spécifiques définis dans notre système. Étant donné que notre papier se
concentre uniquement sur les libraires \emph{Keras}, \emph{Tensorflow} et \emph{Pytorch}, nous avons utilisé les mots
clés présentés dans le tableau \ref*{tab:keywords}. Ces mots clés
représentent les modules et les fonctions de ces deux librairies.\\


\begin{table}[h]
  \centering
  \caption{\emph{Liste des mots clés utilisés pour la recherche de programmes d'apprentissage profond dans les dépôts Github.}}
  \label{tab:keywords}
  \begin{tabular}{|l|l|}
    \hline
    \textbf{Mots clés}         & \textbf{Librairie} \\ \hline
    keras.layers               & Keras              \\ \hline
    keras.layers.convolutional & Keras              \\ \hline
    AveragePooling2D           & Keras/Tensorflow   \\ \hline
    MaxPooling2D               & Keras/Tensorflow   \\ \hline
    tensorflow.keras.layers    & Tensorflow         \\ \hline
    Conv2D                     & Keras/Tensorflow   \\ \hline
    Convolution2D              & Keras/Tensorflow   \\ \hline
    BatchNormalization         & Keras/Tensorflow   \\ \hline
    import torch               & Pytorch            \\ \hline
    import torchvision.models  & Pytorch            \\ \hline
    torch.nn.Sequential        & Pytorch            \\ \hline
    torch.nn.Conv2d            & Pytorch            \\ \hline
    torch.nn.BatchNorm2d       & Pytorch            \\ \hline
    torch.nn.MaxPool2d         & Pytorch            \\ \hline
  \end{tabular}
\end{table}


La requête de recherche de type \emph{search code} retourne un ensemble d'informations sur le repository et le fichier contenant les mots clés recherchés.
Nous procédons ensuite à un filtrage des répertoires selon les critères
suivants: (1) nombre de commit, (2) notre de star, (3) nombre de fork, (4) dernière date du
commit, (5) nombre de contributeurs. Ce filtrage nous permet de ne garder que
les répertoires qui sont les plus populaires et qui sont les plus actifs. Nous
avons ensuite procédé à un filtrage manuel des répertoires en éliminant les
projets qui ne sont pas des programmes d'apprentissage profond.








\subsection{Data Processing}
\label{sec:Data Processing}
In this sub-section, we describe the details of the data collection and analysis approach followed to answer our different research questions. This approach is depicted in Figure \ref{fig:1}. In the following, we elaborate on each data processing step.



\begin{enumerate}
  % \item \emph{Creation d'un code de simulation de déploiment avec tebderly:}(https://docs.tenderly.co/simulations-and-forks/simulation-api) ou ganache (https://www.youtube.com/watch?v=fH4KUgQCS7c&ab_channel=BenBK)

  \item \emph{Collecter et trier les Smart contract:}

  \item \emph{Mesurer le prix du gaz de chaque Smart contrat:}
        https://github.com/paperSubmition2020/GasmetReplicationPackage


  \item \emph{Inserer des modifications (aléatoirement - manuellement):}

  \item \emph{Documenter les modifications:}


  \item \emph{Mesurer le prix du gaz des smart contrats modifés:}

\end{enumerate}




\subsection{Replication Package}
\label{sec:Replication Package}
